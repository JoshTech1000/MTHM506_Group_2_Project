\section{Critical Review and Conclusion}
Drawbacks of the Model:
\\
1. Predictions do not cover full range of data, as evinced by deviations in the QQ plot
\\

\section{Conclusions}
Firstly, based on the Correlogram we can conclude that no single socio-economic covariate has much linear correlation with TB incidence, but illiteracy, urbanisation, poverty, sanitation, unemployment and timeliness of notification are all weakly correlated with TB incidence, and given that there are more strong correlations between these socio-economic covariates, an increase in illiteracy, poverty, unemployment and Increases in illiteracy, poverty, unemployment and poor sanitation will simultaneously lead to decreases in urbanisation and timeliness of notification, ultimately leading to significant increases in TB incidence. Poverty is strongly correlated with several socio-economic covariates and is the primary factor that governments need to improve. Sanitation and urbanisation are also more strongly correlated, implying that good urban infrastructure and quality health resources have a greater impact on reducing TB incidence. According to our predicted TB incidence map, Brazil's central region has a lower incidence overall, so we recommend that the health sector invest significant health resources to improve the current situation throughout Brazil's north-west, followed by localised areas in the south and east, although these areas could also rely on assistance from neighbouring regions with fewer cases to improve their situation.